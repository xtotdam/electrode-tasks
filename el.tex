\input{header.tex}

\newcommand{\num}[1]{\line(1,0){500}\\\textbf{#1}\\}
\fancyhf{}
\rfoot{\hfill Xtotdam, \today,\;\currenttime}
\cfoot{\thepage}
\lfoot{'*' means ``to be added soon''}

\begin{document}
Электродинамика - задачи\\

\num{3.2}
$\triangle \varphi = -4\pi \rho$\\
$\triangle f(\vec{r}) = \dfrac1{r^2} \pd{}{r} (r^2 \pd{}{r}f(\vec{r})) $\\
$\triangle \dfrac{A}{r} = A \delta(r) $\\

\num{4.1,\,4.4}
$\varphi_0 = \dfrac{Q}{r};\; \varphi_1 = \dfrac{\scmult{d}{r}}{r^3};\;\varphi_2 = \sum\limits_{\alpha\beta}^3 \dfrac{D_{\alpha\beta}x^\alpha x^\beta}{2r^5}$\\
$Q = \int dV' \rho(\vec{r'});\;\vec{d}=\int dV' \vec{r'}\rho(\vec{r'});\; D_{\alpha\beta} = \int dV' \{3x^\alpha x^\beta - r'^2 \delta_{\alpha\beta}\}\rho(\vec{r'})  $\\

\num{4.3}
$\rho(\vec{r'}) =
\begin{cases}
    \sigma_S\delta(z'), & \vec{r'} \le R \\
    0, & \text{else}
\end{cases}$\\

\num{4.5}
$\rho(\vec{r'}) = \dfrac{q}{2\pi}\delta{z'}\left(\dfrac{\delta(r'-a)}{a}-\dfrac{\delta(r'-b)}{b} \right)$\\

\num{5.2}
$\varepsilon_{int} = q_1 \varphi_2(\vec{r_1}) - \scmult{d_1}{E_2(\vec{r_1})};\;q_1 = 0;\; \vec{E} = \dfrac{3\scmult{p}{r}\vec{r}-\vec{p}r^2}{r^5}$\\
$\vec{F_{int}} = -\vec{\nabla}\varepsilon_{int}$\\

\num{5.4}
$\vec{A} = \vec{A_1} = \dfrac{\vmult{m}{r}}{r^3};\;\vec{m} = \dfrac1{2c}\int dV' \vmult{r'}{j(\vec{r'})} $\\
$ \vec{j(\vec{r'})} = \rho(\vec{r'}) \cdot \vec{v(\vec{r'})};\; \rho(\vec{r'}) = \dfrac{q}{V} = \begin{cases}\dfrac{3q}{4\pi R^3} & r'\le R \\ 0 & else \end{cases};\; v(\vec{r'}) = \vmult{\omega}{r'}$\\
$\vec{H} = \dfrac{3\scmult{m}{r}\vec{r}-\vec{m}r^2}{r^5}$\\

\num{7.2}
$r'(t) = R(\cos \omega t \vec{e_x} + \sin \omega t \vec{e_y})$\\
$\d{I}{\Omega}=\dfrac{\vmult{\ddot{d}}{n}}{4\pi c^3};\; I = \dfrac{2\ddot{\vec{d}}^2(\tau)}{3c^3};\; \vec{d} = \vec{d(\tau)}! $\\

\num{7.3}
$\d{I}{\Omega}=\dfrac{\vmult{\ddot{m}}{n}}{4\pi c^3};\; I = \dfrac{2\ddot{\vec{m}}^2(\tau)}{3c^3}$\\
$\vec{m} = \dfrac{JS\vec{N}}{c} = \dfrac{J\pi a^2}{c}\vec{N};\; \vec{N} = \{\cos\omega\tau \sin\alpha,\,\sin\omega\tau\sin\alpha,\,\cos\alpha \}$\\

\num{7.5}
$\rho(\vec{r'},\tau) = A \delta(z')\delta(r'-R)\left(\delta(\varphi'-\omega\tau)+\delta(\varphi'-\omega\tau-\pi) \right);\; A = \dfrac{q}{R}$\\
$I = \sum\limits_{\alpha\beta}^3 \dfrac{\dddot{D}_{\alpha\beta} \dddot{D}_{\alpha\beta}}{180c^5};\;\vec{H}=\dfrac{\vmult{\dddot{D}}{n}}{6c^3 r} $\\

\num{7.7}
$\rho = q \delta(x) \delta(y) \left( \delta(z-l-a\cos{\omega t}) - \delta(z+l-a\cos{\omega t})) \right)$\\
$\vec{d} = 0;\; \vec{m} = 0;\; \Omega \Leftarrow D(\omega);\; D_{11}=D_{22}=-\dfrac1{2}D_{33};\;D_{12}=D_{23}=D_{31}=0$\\
$I = \sum\limits_{\alpha\beta}^3 \dfrac{\dddot{D}_{\alpha\beta} \dddot{D}_{\alpha\beta}}{180c^5};\; \varepsilon_T = \int\limits_0^T I dt;\; \int\limits_0^T \sin^2 \omega t dt = \dfrac{1}{2}$\\

\num{8.2}
$\triangle \varepsilon = \int\limits_{-\infty}^\infty I(t)dt;\; I(t) = \dfrac{2\ddot{\vec{d(t)}}}{3c^3}; \vec{d}=e\vec{r};\;m\ddot{\vec{r}}=\dfrac{eQ\vec{r}}{r^3};\;r^2(t)=v_0^2 t^2 + a^2$\\
$\int\limits_{-\infty}^\infty \dfrac{d\xi}{(\xi^2+a^2)^2} = \dfrac{\pi}{a}$\\

\num{8.3}
$\d{\varepsilon}{t} = -I = -\dfrac{2\ddot{\vec{d}}^2}{3c^3} = -\dfrac{2e\ddot{\vec{r}}^2}{3c^3};\;\vec{r}=\vec{r_\perp}+\vec{r_\|};\;m\ddot{\vec{r}}=\dfrac{e}{c}\vmult{\dot{r}}{H}$\\
$\d{\varepsilon}{t} = \d{}{t}\left(\dfrac{mv^2}{2} \right);\; \d{(\ddot{r}_\|^2)}{t} = 0$\\

\num{11.6}
$* $\\

\num{11.8}
$x^\alpha = \Lambda^\alpha_\beta {x'}^\beta ;\; \Lambda^\alpha_\beta =
\begin{pmatrix}\gamma & \beta\gamma & 0 & 0 \\ \beta\gamma & \gamma & 0 & 0 \\ 0 & 0 & 1 & 0 \\ 0 & 0 & 0 & 1 \end{pmatrix};\;\gamma = \dfrac1{\sqrt{1-\beta^2}};\;\beta=\dfrac{v}{c};\;g_{ik}=(\,+\,-\,-\,-\,)$\\
$A^i = \{\varphi, \vec{A} \};\; \vec{A}'=0\; (v_q\, in\, K' = 0) $\\

\num{13.3}
$\begin{cases}
    \vec{E_\|}'=\vec{E_\|} & \vec{E_\perp}' = \gamma(\vec{E_\perp}+\dfrac1{c}\vmult{V}{H} ) \\
    \vec{H_\|}'=\vec{H_\|} & \vec{H_\perp}' = \gamma(\vec{H_\perp}-\dfrac1{c}\vmult{V}{E} )
\end{cases}$\\
$\vmult{E'}{H'}=0$\\

\num{14.3}

\num{14.4}
$\begin{cases}
    \varepsilon_f^0 + mc^2 = \varepsilon_f + \varepsilon_e \\
    \vec{p}_f^0 = \vec{p}_e + \vec{p}_f
\end{cases};\;
\begin{cases}
    \dfrac{{\varepsilon_f^0}^2}{c^2}-{p^0_f}^2 = 0 \\
    \dfrac{{\varepsilon_e}^2}{c^2}-{p_e}^2 = m^2 c^2
\end{cases};\;
\begin{cases}
    \varepsilon^0_f = \hbar\omega_0 \\
    \varepsilon_f = \hbar\omega \\
\end{cases}\;\Rightarrow\; \omega = \dfrac{\omega_0}{1+\dfrac{\hbar\omega_0}{mc^2}(1-\cos\theta)}$

\num{14.6}
$* $\\

\num{14.8}
$ $\\

\num{16.1}
$* $\\

\num{16.2}
$ $\\

\num{19.1}
$ $\\

\num{19.4}
$ $\\

\num{19.5}
$ $\\

\num{21.2}
$ $\\

\num{21.3}
$ $\\

\num{22.2}
$\begin{cases}E^I_\tau = E^{II}_\tau\\ D^I_n = D^{II}_n\end{cases};\;\vec{D}=\varepsilon\vec{E};\;
\begin{cases}\varphi_I=\dfrac{q}{\varepsilon_1 r_1}+\dfrac{q'}{r_2}\\\varphi_{II}=\dfrac{q''}{r_1}\end{cases}$\\
$\vec{F}_q = q\vec{E}_\text{возм.}|_{x=y=0,\,z=a}$\\

\num{22.3}
$\varphi = \varphi_\text{внешн.}+\varphi_\text{возм.};\;
\begin{cases}x=r\cos\varphi\sin\theta\\ y=r\sin\varphi\sin\theta\\ z=r\cos\theta\end{cases};\;
\text{ГУ: }\begin{cases}\varphi_I=\varphi_{II}|_{r=R}\\ \varepsilon_1\pd{\varphi_I}{r}=\varepsilon_2\pd{\varphi_{II}}{r}|_{r=R}\end{cases}$\\

\num{24.4}
$R=\dfrac{U}{I}=\dfrac{\abs{\varphi_1-\varphi_2}}{I};\; \varphi(\vec{r})=\dfrac{A}{r}+\dfrac{B}{\abs{\vec{r}-\vec{L}}};\;
\begin{cases}\varphi_1=\varphi(\vec{r}=\vec{a})\\\varphi_2=\varphi(\vec{r}-\vec{L}=\vec{b})\end{cases}$\\
$\vec{j}=\sigma\vec{E}=-\sigma\vec{\nabla}\varphi(\vec{r});\;I_1=-I_2=I=\int_{S_1}\scmult{j}{dS};\;R=\dfrac1{4\pi\sigma}\left(\dfrac1{a}-\dfrac1{b}\right)  $\\

\num{24.5}
$\rot{H} = \dfrac{4\pi}{c}\vec{j};\; \oint\vec{H}\vec{dl}=\dfrac{4\pi}{c}\int\vec{j}\vec{dS} $\\
$\int\limits_0^{2\pi}H(r)rd\varphi = \dfrac{4\pi}{c} \dfrac{I}{\pi R^2} \int\limits_0^r rdr \int\limits_0^{2\pi} d\varphi$\\


\num{26.2}
$\varepsilon_{int}=\dfrac1{c^2}L_{12}I_1 I_2;\;\Phi_2 = \int\limits_{S_2}\scmult{B_1}{dS_2}=\dfrac{L_{12}I_1}{c};\;B_1=\dfrac{2I_1}{cr}$\\

\num{26.3}
$\varepsilon_M=\int\dfrac{\vec{H}^2}{8\pi}dV=\dfrac{LI^2}{2c^2} \Rightarrow L=\dfrac{2c^2\varepsilon_M}{I^2}$\\
$\oint\scmult{H}{dl}=\int\limits_0^{2\pi}H(r)rd\varphi=2\pi rH(r)=\dfrac{4\pi}{c}
\begin{cases} 0 & r<R,\,r>R+a\\ NJ & R\le r\le R+a \end{cases}$\\

\num{26.5}
$L_{12}=\mu\oint\limits_{L_1}\oint\limits_{L_2}\dfrac{\scmult{dl_1}{dl_2}}{\abs{\vec{r}_1-\vec{r}_2}}$\\
$\text{cheat:\qquad}L_{12}=\dfrac{2\mu(\pi ab)^2 a}{h^3},\;h\gg a,\,h\gg b$\\

\num{27.1}
$* $\\

\num{27.3}
$* $\\

\num{28.1}
$* $\\

\num{30.1}
$\begin{cases}
\varepsilon'(\omega)=1+\dfrac1{\pi} \text{v.p.\!\!\!\!}\int\limits_{-\infty}^\infty \dfrac{\varepsilon''(\omega)}{x-\omega}dx\\
\varepsilon''(\omega)=-\dfrac1{\pi} \text{v.p.\!\!\!\!}\int\limits_{-\infty}^\infty \dfrac{\varepsilon'(\omega)-1}{x-\omega}dx
\end{cases}$\\
$\text{cheat:\qquad v.p.\!}\int\limits_{-\infty}^\infty \dfrac{xdx}{(x-\omega)(x^2+\nu^2)} = \dfrac{\pi\nu}{\omega^2+\nu^2}$\\


\end{document}
