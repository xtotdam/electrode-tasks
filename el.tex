\documentclass[a4paper,12pt]{report}
\usepackage[T2A]{fontenc}
\usepackage[utf8]{inputenc}
\usepackage{amssymb,amsfonts}
\usepackage[fleqn]{amsmath}
\usepackage[russian,english]{babel}
\usepackage{anyfontsize}
\usepackage{graphicx}
\usepackage{fancyhdr}
\usepackage{cmap}
% \usepackage{times}
\usepackage{mathrsfs}
\usepackage[inline]{enumitem}
\usepackage[usenames,dvipsnames,svgnames,table]{xcolor}
\usepackage[ddmmyyyy,hhmmss]{datetime}

\setlength{\headheight}{15.2pt}
\renewcommand{\headrulewidth}{0pt}
\pagestyle{fancy}

\usepackage{geometry}
\geometry{left=2cm}
\geometry{right=1.5cm}
\geometry{top=1cm}
\geometry{bottom=2cm}

\let\divsymb=\div % rename builtin command \div to \divsymb
\let\underdot=\d % rename builtin command \d{} to \underdot{}

\DeclareMathOperator{\sinc}{sinc}
\DeclareMathOperator{\const}{const}
\DeclareMathOperator{\Tr}{Tr}   %trace
\DeclareMathOperator{\Dim}{dim} %dimensions

\newcommand{\abs}[1]{\left| #1 \right|} % for absolute value
\newcommand{\norm}[1]{\lVert #1 \rVert} % for norm ||f||
\newcommand{\qnorm}[1]{\lVert #1 \rVert ^2} % for norm quadrat ||f||^2
\newcommand{\avg}[1]{\left< #1 \right>} % for average

\newcommand{\vmult}[2]{[\vec{#1}{\times}\vec{#2}]} % vector multiplication
\newcommand{\scmult}[2]{(\vec{#1}{\cdot}\vec{#2})} % scalar multiplication

\renewcommand{\d}[2]{\dfrac{d #1}{d #2}} % for derivatives
\newcommand{\dd}[2]{\dfrac{d^2 #1}{d #2^2}} % for double derivatives
\newcommand{\pd}[2]{\dfrac{\partial #1}{\partial #2}} % for partial derivatives
\newcommand{\pdd}[2]{\dfrac{\partial^2 #1}{\partial #2^2}} % for double partial derivatives
\newcommand{\pddv}[3]{\dfrac{\partial^2 #1}{\partial #2 \partial #3}} % for double partial derivatives, diff. variables
\newcommand{\pdc}[3]{\left( \dfrac{\partial #1}{\partial #2} \right)_{#3}} % for thermodynamic partial derivatives

\newcommand{\ket}[1]{\left| #1 \right>} % for Dirac bras
\newcommand{\bra}[1]{\left< #1 \right|} % for Dirac kets
\newcommand{\braket}[2]{\left< #1 \vphantom{#2} \right|\left. #2 \vphantom{#1} \right>} % for Dirac brackets
\newcommand{\matrixel}[3]{\left< #1 \vphantom{#2#3} \right| #2 \left| #3 \vphantom{#1#2} \right>} % for Dirac matrix elements

\newcommand{\comml}[2]{\left[ \hat{#1}, #2 \right]} % [A,...]
\newcommand{\commr}[2]{\left[ #1, \hat{#2} \right]} % [...,B]
\newcommand{\comm}[2]{\left[ \hat{#1}, \hat{#2} \right]} % for commutator for only two operators like [A,B]

\renewcommand{\div}[1]{\operatorname{div}\vec{#1}} % dirty hack for divergence
\newcommand{\rot}[1]{\operatorname{rot}\vec{#1}}
\newcommand{\grad}[1]{\operatorname{grad}\vec{#1}}

\newcommand{\gradn}[1]{\vec{\nabla} #1} % for gradient
\newcommand{\divn}[1]{\scalmult{\nabla}{#1}} % for divergence
\newcommand{\rotn}[1]{\vecmult{\nabla}{#1}} % for curl

\newcommand{\praiseme}[1]{\scalebox{.4}{ \textcircled{\scalebox{.5}{CC}} \textcircled{\scalebox{.5}{BY}} \textcircled{\scalebox{.5}{SA}} #1, \the\day.\the\month.\the\year}}
\newcommand{\tobewritten}[0]{\textcolor{red}{\textsc{To be written}}}
\newcommand{\NB}[0]{\textbf{NB!}}
\newcommand{\Def}[0]{$\mathfrak{Def.}$}
\newcommand{\Th}[0]{$\mathfrak{[Th.]}$}
\newcommand{\Wiki}[0]{$\mathfrak{(Wiki)}$}

\newcommand{\bbar}[1]{\bar{\bar{#1}}}

\makeatletter
\newcommand{\inlineitem}[1][]{
    \ifnum\enit@type=\tw@
        {\descriptionlabel{#1}}
        \hspace{\labelsep}
    \else
        \ifnum\enit@type=\z@
        \refstepcounter{\@listctr}\fi
    \quad\@itemlabel\hspace{\labelsep}%
    \fi
}
\makeatother


\newcommand{\num}[1]{\line(1,0){500}\\\textbf{#1}\\}

\begin{document}
Электродинамика - задачи\\

\num{3.2}
$\triangle \varphi = -4\pi \rho$\\
$\triangle f(\vec{r}) = \dfrac1{r^2} \pd{}{r} (r^2 \pd{}{r}f(\vec{r})) $\\
$\triangle \dfrac{A}{r} = A \delta(r) $\\

\num{4.1,\,4.4}
$\varphi_0 = \dfrac{Q}{r};\; \varphi_1 = \dfrac{\scmult{d}{r}}{r^3};\;\varphi_2 = \sum\limits_{\alpha\beta}^3 \dfrac{D_{\alpha\beta}x^\alpha x^\beta}{2r^5}$\\
$Q = \int dV' \rho(\vec{r'});\;\vec{d}=\int dV' \vec{r'}\rho(\vec{r'});\; D_{\alpha\beta} = \int dV' \{3x^\alpha x^\beta - r'^2 \delta_{\alpha\beta}\}\rho(\vec{r'})  $\\

\num{4.3}
$\rho(\vec{r'}) =
\begin{cases}
    \sigma_S\delta(z'), & \vec{r'} \le R \\
    0, & \text{else}
\end{cases}$\\

\num{4.5}
$\rho(\vec{r'}) = \dfrac{q}{2\pi}\delta{z'}\left(\dfrac{\delta(r'-a)}{a}-\dfrac{\delta(r'-b)}{b} \right)$\\

\num{5.2}
$\varepsilon_{int} = q_1 \varphi_2(\vec{r_1}) - \scmult{d_1}{E_2(\vec{r_1})};\;q_1 = 0;\; \vec{E} = \dfrac{3\scmult{p}{r}\vec{r}-\vec{p}r^2}{r^5}$\\
$\vec{F_{int}} = -\vec{\nabla}\varepsilon_{int}$\\

\num{5.4}
$\vec{A} = \vec{A_1} = \dfrac{\vmult{m}{r}}{r^3};\;\vec{m} = \dfrac1{2c}\int dV' \vmult{r'}{j(\vec{r'})} $\\
$ \vec{j(\vec{r'})} = \rho(\vec{r'}) \cdot \vec{v(\vec{r'})};\; \rho(\vec{r'}) = \dfrac{q}{V} = \begin{cases}\dfrac{3q}{4\pi R^3} & r'\le R \\ 0 & else \end{cases};\; v(\vec{r'}) = \vmult{\omega}{r'}$\\
$\vec{H} = \dfrac{3\scmult{m}{r}\vec{r}-\vec{m}r^2}{r^5}$\\

\num{7.2}
$r'(t) = R(\cos \omega t \vec{e_x} + \sin \omega t \vec{e_y})$\\
$\d{I}{\Omega}=\dfrac{\vmult{\ddot{d}}{n}}{4\pi c^3};\; I = \dfrac{2\ddot{\vec{d}}^2(\tau)}{3c^3};\; \vec{d} = \vec{d(\tau)}! $\\

\num{7.3}
$\d{I}{\Omega}=\dfrac{\vmult{\ddot{m}}{n}}{4\pi c^3};\; I = \dfrac{2\ddot{\vec{m}}^2(\tau)}{3c^3}$\\
$\vec{m} = \dfrac{JS\vec{N}}{c} = \dfrac{J\pi a^2}{c}\vec{N};\; \vec{N} = \{\cos\omega\tau \sin\alpha,\,\sin\omega\tau\sin\alpha,\,\cos\alpha \}$\\

\num{7.5}
$\rho(\vec{r'},\tau) = A \delta(z')\delta(r'-R)\left(\delta(\varphi'-\omega\tau)+\delta(\varphi'-\omega\tau-\pi) \right);\; A = \dfrac{q}{R}$\\
$I = \sum\limits_{\alpha\beta}^3 \dfrac{\dddot{D}_{\alpha\beta} \dddot{D}_{\alpha\beta}}{180c^5};\;\vec{H}=\dfrac{\vmult{\dddot{D}}{n}}{6c^3 r} $\\

\num{7.7}
$\rho = q \delta(x) \delta(y) \left( \delta(z-l-a\cos{\omega t}) - \delta(z+l-a\cos{\omega t})) \right)$\\
$\vec{d} = 0;\; \vec{m} = 0;\; \Omega \Leftarrow D(\omega);\; D_{11}=D_{22}=-\dfrac1{2}D_{33};\;D_{12}=D_{23}=D_{31}=0$\\
$I = \sum\limits_{\alpha\beta}^3 \dfrac{\dddot{D}_{\alpha\beta} \dddot{D}_{\alpha\beta}}{180c^5};\; \varepsilon_T = \int\limits_0^T I dt;\; \int\limits_0^T \sin^2 \omega t dt = \dfrac{1}{2}$\\

\num{8.2}
$\triangle \varepsilon = \int\limits_{-\infty}^\infty I(t)dt;\; I(t) = \dfrac{2\ddot{\vec{d(t)}}}{3c^3}; \vec{d}=e\vec{r};\;m\ddot{\vec{r}}=\dfrac{eQ\vec{r}}{r^3};\;r^2(t)=v_0^2 t^2 + a^2$\\
$\int\limits_{-\infty}^\infty \dfrac{d\xi}{(\xi^2+a^2)^2} = \dfrac{\pi}{a}$\\

\num{8.3}

\num{11.6}



\num{11.8}
$x^\alpha = \Lambda^\alpha_\beta {x'}^\beta ;\; \Lambda^\alpha_\beta =
\begin{pmatrix}\gamma & \beta\gamma & 0 & 0 \\ \beta\gamma & \gamma & 0 & 0 \\ 0 & 0 & 1 & 0 \\ 0 & 0 & 0 & 1 \end{pmatrix};\;\gamma = \dfrac1{\sqrt{1-\beta^2}};\;\beta=\dfrac{v}{c};\;g_{ik}=(\,+\,-\,-\,-\,)$\\
$A^i = \{\varphi, \vec{A} \};\; \vec{A}'=0\; (v_q\, in\, K' = 0) $\\

\num{13.3}
$\begin{cases}
    \vec{E_\|}'=\vec{E_\|} & \vec{E_\perp}' = \gamma(\vec{E_\perp}+\dfrac1{c}\vmult{V}{H} ) \\
    \vec{H_\|}'=\vec{H_\|} & \vec{H_\perp}' = \gamma(\vec{H_\perp}-\dfrac1{c}\vmult{V}{E} )
\end{cases}$\\
$\vmult{E'}{H'}=0$\\

\num{14.3}

\num{14.4}
$\begin{cases}
    \varepsilon_f^0 + mc^2 = \varepsilon_f + \varepsilon_e \\
    \vec{p}_f^0 = \vec{p}_e + \vec{p}_f
\end{cases};\;
\begin{cases}
    \dfrac{{\varepsilon_f^0}^2}{c^2}-{p^0_f}^2 = 0 \\
    \dfrac{{\varepsilon_e}^2}{c^2}-{p_e}^2 = m^2 c^2
\end{cases};\;
\begin{cases}
    \varepsilon^0_f = \hbar\omega_0 \\
    \varepsilon_f = \hbar\omega \\
\end{cases}\;\Rightarrow\; \omega = \dfrac{\omega_0}{1+\dfrac{\hbar\omega_0}{mc^2}(1-\cos\theta)}$

\num{14.6}

\num{14.8}

\num{16.1}

\num{16.2}


\num{19.1}

\num{19.4}

\num{19.5}

\num{21.2}

\num{21.3}

\num{22.2}

\num{22.3}

\num{24.4}

\num{24.5}


\num{26.2}

\num{26.3}

\num{26.5}

\num{27.1}

\num{27.3}

\num{28.1}

\num{30.1}


\end{document}
